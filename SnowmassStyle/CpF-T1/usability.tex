Thanks to sustained advances in hardware and software
technologies, computer modeling is playing an increasingly
important role for particle accelerators, making it logical to
strengthen programmatic activities. Numerous simulation codes
have been developed, and with a few notable exceptions (e.g., the
SciDAC funded collaboration) the development paradigm has largely
been: a code linked to a project or specialized topic, developed
by a researcher (usually a physicist), with occasional help from
computer scientists.  Maximizing scientific output per dollar
means maximizing the usability of the pool of codes while
minimizing spending on development and support, including through
reductions of duplication/increases in modularity and code
interoperability.

Development and application of accelerator algorithms and codes
have become extremely complex and specialized endeavors, calling
for teams including computational physicists (SciDAC but
expand...), applied mathematicians and computer scientists. Such
an approach is being adopted elsewhere, and calls for a higher
level of coordination among modeling and code development
efforts, with progressive integration of codes into a tool set.
This is all the more timely as computer architectures are
transitioning to new technologies, requiring adaptation.
Separation between software for personal computers versus
supercomputers is also diminishing as the former become multicore
and the later commodity based, and it is essential to envision
tools that function well on a broad range of platforms.

A high-level scripting interface for rapid development and
prototyping, offering easy interfacing with high performance
languages and expandability, represents one solution to the
challenge of coupling of existing codes while minimizing
disruption and enabling both interoperability and expansion
capabilities. With such a construct, existing codes continue
unmodified preserving the very significant investments in
existing accelerator modeling tools, while their functionalities
are exposed to allow combined use for multi-physics simulation.
In the past decade, the Python scripting language has emerged as
a high-level solution for rapid development and prototyping which
can be easily coupled to the high-performance programming
languages.

% JRC: I disagree with the above.  We have done/are doing extensive
% multiphysics with Vorpal without the above approach, but instead
% through well-defined C++ APIs.  Further, the above approach with
% shared library loading is known NOT to work well on LCFs.
% I have introduced a compromise, which is to say this is one
% direction, without saying that it gives "unprecedented power".

Current practice is also less than optimal in that with few exceptions
the users of HPC accelerator codes are the developers.  Scientific
productivity would be enhanced by making the accelerator codes more
widely usable.  This include simplified problem setup and submission
through graphical user interfaces with client-server technology, as
has already occured in, e.g., the HPC visualization community.

