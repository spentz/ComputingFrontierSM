Conventional accelerator technologies play an important role in
the design of future accelerators both in the Energy Frontier
(EF) and the Intensity Frontier (IF). These technologies, which
have been proven to work in existing accelerators, include the
normal conducting rf and superconducting rf acceleration schemes.
Electromagnetic simulations of accelerator components and systems
are essential to the design and optimization of these machines.
In particular, virtual prototyping of accelerator components and
systems through high performance computing enables accelerator
builders to shorten the time for the design and build cycle,
which will substantially reduce the cost for achieving an
optimized design that satisfies beam quality preservation and
machine operational reliability. These machines include

\begin{itemize}
 \item A high-intensity proton source based on superconducting rf
   technology (IF)
 \item A linear electron-positron collider based on superconducting
   rf technology, capable of delivering 500 GeV -- 1 TeV center of
   mass energy (EF)
 \item A linear electron-positron collider based on high-gradient
   normal conducting rf technology and two beam acceleration techniques,
   capable of delivering 500 GeV -- 3 TeV center of mass energy (EF).
   The computational issues in electromagnetic modeling and simulation
   related to these machines are as follows.
\end{itemize}

For superconducting rf technology that is used in the linacs of
proposed accelerators such as Project X and ILC, the accelerator
cavity is designed to minimize the effects of high-order-modes
(HOMs) to maintain beam stability and to limit extra heat losses
on cryogenics. However, during the fabrication process, the SRF
cavity is deformed from its designed shape because of loose
machining tolerance and the tuning procedure, the HOM properties
such as their external Q can be substantially changed to cause
beam breakup problems at the currents well below the
designed threshold. Furthermore, misalignments of the cavities in
a cryomodule will affect the wakefield even though the
imperfection effects in a single cavity is well understood.
Simulation using the capacity of supercomputers will be an
invaluable tool to study these effects and will provide insights
of how to mitigate any possible problems.  In addition, the
 statistical analysis for a wide range of the scales and
types of deformation and misalignments in these structures
require thousands of computers runs to give a reliable account of
wakefield effects during machine operation.

Another limiting factor that prevents the accelerator from reaching
high gradients is the generation of dark current, which arises
from field emissions of electrons from the surface of an
accelerating structure and their subsequent movement whose
trajectories are determined by the accelerating rf field. Dark
current may lead to beam loading of the accelerator structure
and, if captured, may also produce undesirable backgrounds
downstream in the detector at the interaction point. Therefore,
understanding the mechanism of dark current generation and
capture is essential to the successful development of high
gradient structures for linear colliders such as the CLIC. Also,
it was found experimentally that dark current generation was
enhanced during the transient of the drive power pulse.
Therefore, it is important to perform a time-domain simulation
with a realistic driving pulse to determine the dark current
effects. The capture of dark current downstream may take a long
distance that may involve multiple accelerating modules. The
number of time steps and the number of particles for tracking
needed for these large-scale simulations requires tens of
millions CPU hours on state-of-the-art supercomputers.

In addition to electromagnetic properties, the
studies of thermal and mechanical properties are necessary for
the full design of a cavity. One first calculates the
electromagnetic properties of an accelerating mode in the cavity,
and then uses them to determine the thermal and/or mechanical
properties, which may be used to evaluate the changes in
electromagnetic properties due to thermal expansion or mechanical
deformation. For the integrated simulation of cavities, the study
of these multi-physics effects is a computationally challenging
problem due to the complexity of the cavity geometry and the
different types of physics. The complexity of the modeling arises from 
the fully-dressed SC
cavity together with fast tuner, slow tuner, rf coupler and
helium vessel, as well as the connections to cavity string
installation inside the cryomodule. In addition to improving
existing thermal solvers to handle various boundary conditions
form the external thermal loads, new parallel mechanical solvers
are needed to address important effects such as microphonics. This
new development will provide a transformative tool that can
facilitate a full design optimization of the machine including
all the details and complexities that are involved in the system.



