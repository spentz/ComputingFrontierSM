\subsection{Dielectric structures}

Dielectric structures have been found promising in both the GHz
and optical wavelengths, with both types of structures relying on
photonic structure principles of frequency-selective confinement.
In the GHz range, photonic structures formed from arrays of
dielectric rods have been found with high Q values for the
accelerating mode but with reduced wakefields due to the lack of
confinement of higher-order modes.  In the optical wavelengths,
the dielectric breakdown field is in the range of 10 or more
GeV/m, and so hold out the promise of acceleration gradients that
are two orders of magnitude greater than conventional systems.
In the optical, structures vary from 3D, such as the woodpile, to
dielectric fibers, which are cheap, as they are used by the
telecom industry.

With many principles and ideas of using dielectric structures
having been elucidated, there is now a need for assessing many
practical issues.  These include issues of pure electromagnetics,
such as how to efficiently couple energy into these structures
and what structures have sufficient Q values, through
self-consistent effects, such as whether there are instabilities
due to wake fields.  For such studies, algorithmic advances are
needed, with one direction being the need for rapid geometry
layout and meshing algorithms for these complex structures, as
well as fast, scalable, and accurate algorithms able to compute
$>10^9$ degrees of freedom.  As well, algorithms need modification
to take advantage of the many computational accelerators and
advanced instructions (GPU, MIC, AVX2) now or soon available.
Moreover, it is important to develop this software in a
maintainable fashion, which cannot be writing different
implementations for each new architecture.

As well, there is a need for integration of optimization to find
the systems with the best coupling, highest Q, lowest wake
fields, etc.  Such optimizers need to be tailored to the type of
simulations.  For example, optimizers based on differentiation
may not work well with some simulations that have significant
error or particle noise.

As the field progresses, there will be a need for multi-physics.
Because the electromagnetic field deforms the structures, there
is a need for electro-acoustic couplings, and because it heats
the structure, electro-thermal coupling is additionally needed.

With the above developing tool suites, there is a need finally to
carry out the extensive studies of these systems.  There are many
configurations now (RF, optical; 3D woodpiles, gratings; 2D rods,
fibers) with many parameters to vary.  Optimization campaigns are
needed, but they cannot be done blindly.  With so many
parameters, physical intuition will also be important.  Hence,
there will need to be a partnership among computational
physicists, algorithm developers, and computer scientists to
bring the promise of this field to fruition.

